\documentclass{article}
\usepackage[heading=true]{ctex}
\usepackage{fancyhdr} %页眉页脚
\usepackage{graphicx}
\usepackage[hidelinks]{hyperref}%目录

\begin{document}

\begin{titlepage}

    \begin{center}

        \includegraphics[width=1\textwidth]{logo.png}\\[1cm]%插入学校logo

\songti      \zihao{3}{\textbf{唯物辩证法对生产生活的指导}}\\%论文题目

    \end{center}
    ~\\
    \[
        
    \]
    
    
    ~\\
    ~\\
    ~\\
\begin{center}
    
    \heiti \zihao{4}
\begin{tabular}{l}

\textbf{姓\quad  名:}\songti\textbf{郭少华}\\
\textbf{学\quad  号:}\songti\textbf{2021050264}\\
\textbf{学\quad  院:}\songti\textbf{数理学院}\\
\textbf{专\quad  业:}\songti\textbf{数学与应用数学}\\

\end{tabular}

\end{center}
~\\
    \begin{center}
        \songti \zihao{4} \textbf{2022年12月}%日期
    \end{center}

\end{titlepage}%完成封面


\tableofcontents


\newpage

\addcontentsline{toc}{section}{摘要}
\zihao{4}
\begin{center}%摘要
\section*{\heiti{摘\quad 要}}
\end{center}%结束摘要
\hspace{8mm}
\songti{
本文研究了唯物辩证法的起源和发展,辩证法起源于古希腊哲学家的讨论,在智者学派、苏格拉底、柏拉图
等人的讨论中与研究中不断向前发展。马克思和恩格斯在创立唯物辩证法时从德国古典主义哲学中汲取了
大量营养,唯物辩证法是最主要理论来源是黑格尔的辩证法。\par
本文还讨论了唯物辩证法的基本特征和主要规律,并将他们在生产生活中的应用做了粗略的分析,
本文期望通过文章的阅读,读者能对唯物辩证法有一个初步的认识,能在生活中自觉地应用这一强
大的思想武器。
}

~\\
~\\
\heiti{关键词}:\songti{唯物辩证法;古希腊哲学;辩证法的应用}\\% 关键词3——5个   

\newpage

\zihao{-4}\songti%正文开始
\setlength{\baselineskip}{20pt}

    \section{绪论}
    唯物辩证法是科学的思想武器,我们党运用唯物辩证法在各项事业中取得了巨大的成功,
    然而很多大学生却对唯物辩证法敬而远之,不了解唯物辩证法,更不懂得运用唯物辩证法
    指导自己的学习生活实践,着实可惜。唯物辩证法经过了实践的检验,是科学的,成熟的
    思想武器,如果大学生对学习一些唯物辩证法,自觉地将唯物辩证法运用到学习实践当中,
    对大学生的成长成才会有莫大的帮助。\par
    本文通过对唯物辩证法在生活实践中的运用的一些探讨,希望帮助大学生建立起对唯物辩证法
    的初步认识,不再将它看做高高在上的哲学理论敬而远之,开始接受它,学习它,自觉地将
    唯物辩证法运用到学习实践当中。\par


\section{唯物辩证法的起源和发展}

    马克思主义哲学提出的唯物辩证法,是辩证法和唯物论的集大成者,在此我们梳理一下唯物辩证法
的起源和发展。\par
\subsection{古希腊时期的辩证法}
    唯物辩证法的思想深深根植于西方的哲学传统。恩格斯在《反社林论》中提到,“古希腊的哲学家
    都是天生的、自发的辩证论者。”\cite{恩格斯}古希腊哲学家赫拉克利特提出,“一切都存在而
    又不存在,因为一切都在流动,都在不断地变化、不断地生成和消逝。”而辩证法就是
    一种用运动、联系、变化、发展的眼光看待事物发展的思维方式。古希腊哲学家做了大量关于辩
    证法的讨论,这成为唯物辩证法的主要思想渊源之一。\par
    古希腊智者学派的思想具有朴素辩证法的思想因素。普罗泰戈拉提出每一个问题都有互相对立的
    方面,用朴素辩证法的思维推动了论辩技艺的发展。高尔吉亚通过对"不存在"的三个论证,驳斥了
    爱利亚学派否认非存在的形而上学观点,揭示了思维与存在的差别,提出了存在与非存在的联
    系与转化,反映了他们思想中的朴素辨证法因素。\par
    据现有资料证明,最早自觉提及‘辩证法’这个范畴并将其思想进行较为完整论述的是苏格拉底。
    \cite{王芳恒}在《理想国》中,苏格拉底关于正义的对话,强调了要超越感性的认识,达到普遍真理
    的高度。他的学生柏拉图在其老师基础上进一步提出,认为辩证法就成为从假设上升到绝对原理
    的唯一方法,建立起以理念为核心的哲学理论体系。\par
    亚里士多德继承了苏格拉底和柏拉图的辩证法思想,强调从普遍接受的意见出发,从而达到对真
    理的认识,将辩证推理推上新的高度。
\subsection{德国古典哲学的辩证法}
    德国古典哲学的辩证法对唯物辩证法的产生有极为深远的影响,是唯物辩证法的主要理论渊源之一。
    在时代的影响下,德国古典哲学家将辩证法的研究推向了一个理论高峰,辩证法在这个时期获得极
    大的发展。\par
    康德是德国古典哲学时期的重要代表人物。有人提出,要想真正理解马克思主义哲学,就必须要
    经过康德。\cite{俞吾金}康德提出了人类的理性有形而上学的先天倾向,
    也就是人类理性先天的从经验证明的那些原理出发,一直上升到超越经验本身的
    问题,“它发现自身只好适用那些超出一切可能经验的范围以外的原理,同时那
    些原理看起来是如此无可非议。”康德的学说被称为“先验辩证法”,开始打破形而上学的思维模式,
    倡导人们要辩证地思考与理性的反思。\par
    黑格尔在18世纪末至19世纪初的哲学中提出了辩证发展的理论。他发现了世界整体的内在联系,
    强调事物是不断变化发展的。他还强调事物发展过程内部的矛盾是事物自身运动和发展的源泉。
    黑格尔的辩证法是唯物辩证法的最主要理论来源。
    \begin{center}
\section{唯物辩证法的基本特征与基本规律}
    \end{center}
    \subsection{两大基本特征}
    \subsubsection{世界是普遍联系的}
    唯物辩证法用普遍联系的观点看待世界和历史,唯物辩证法指出:世界是一个有机的整体,世界
    上的一切事物都处于相互影响、相互作用、相互制约之中,反对以片面或孤立的观点看问题。
    \subsubsection{世界是永恒发展的}
    唯物辩证法指出:世界是一个过程,过程是由状态组成的,状态是过程中的状态;
    世界上没有永恒的事物,有生必有灭,无灭必无生;旧事物灭亡的同时,就意味着新事物的产生。
    \subsection{三大基本规律}
        \subsubsection{对立统一规律}

    对立统一规律即事物的矛盾规律,揭示了事物联系的根本内容和发展的动力,是唯物辩证法
    的实质和核心。
        \subsubsection{质量互变规律}

    事物的质、量、度。任何事物都具有质和量这两种规定性。质是指一事物区别于它事物的内部的规定
    性。量是指事物存在和发展的规律。一定的事物都具有一定的质和一定的量,是质和量的统一体。质
    与事物是直接同一的,一定的质就是一定的事物。

    质是人们区分、认识具体事物的客观依据。量与事物不是直接同一的。事物在一定范围内量的变
    化,只要不引起质的变化,一事物仍保持其质的稳定性,仍是原来的事物。这种保持事物质的数
    量界限就是度。度是质和量的统一。

    \subsubsection{否定之否定规律}
    
    否定之否定规律.肯定—否定—否定之否定,是事物矛盾运动的进一步展开,
    它所提示的是事物发展的道路和总趋势。

    辩证法规律揭示的全是极限本质之间的联系,是抽象程度最高的产物。
    尽管辩证法的规律都是从概念的推演中抽象出来的,但是这些规律完全与
    客观现实的本质运动相一致,因此它们都是具有极限真理的客观规律。  
\section{唯物辩证法如何指导人们活动}
    \subsection{辩证发展观指导下的生产生活活动}

    唯物辩证法指出,世界是永恒发展的。不管是人类社会还是对于具体的人类个体,发展是永恒
    的话题。习近平总书记指出,正确的发展实践需要正确的理论指导。发展理念的正确与否,直接
    关系到实践进程的快慢和实践活动的成败。本节探究辩证法的发展观如何正确指导人们的生产实践。\par
        \subsubsection{创新是发展的第一动力}
        习近平总书记指出:“谁抓住了创新,就抓住了牵动经济社会发展全局的‘牛鼻子’。”由此
        可见创新对于一个社会经济快速稳定发展的重要性。近年来我国关键技术领域屡屡受制于人,
        处处收到“制裁”“打压”,直接对我国的经济、科技发展造成极大的阻力和障碍。究其原因,
        是因为我国的创新能力不足,创新人才紧缺,在关键技术领域的研发尚未取得突破性进展,这
        便是创新能力不足对社会经济发展的掣肘作用。如果我们研究一下世界经济的历史便不难发现,
        英国抓住了第一次工业革命的机遇,率先在全世界实现机械化生产,一度造就了“日不落帝国”的
        神话。美国抓住第二次世界大战的契机,大力培养创新型人才,大力发展科学技术,成就了美国
        “超级大国”的大国地位。\par
        由此可见,谁抓住了创新,谁就掌握了发展的主动权。近年来我国大力发展科教兴国和人才强国战略,
        新发展战略更是将创新摆在极其重要的位置,抓紧从“制造大国”向“智造大国”的转变,并取得了一些
        可喜的成绩。我们在欣喜的同时也应认识到我们同发达国家的差距还很大,建设创新型强国还有很长的
        路要走,坚持创新驱动发展战略必须加快步伐不停歇,争取早日在关键技术领域不再受制于人,成为创
        新型大国、强国,牢牢掌握自主发明权,创造权。\par
        bsection{新旧事物交替运动推动事物向前发展}
        新旧事物的交替,是客观存在的自然规律,不以人的意志而转移。新事物的产生总会受到旧事物
        的反抗、压制。在事物发展过程中,新事物可能暂时被旧事物打败,但从长远的运动趋势来看,新
        事物必定战胜旧事物,推动着事物向前发展。刚提出改革开放时很多人不理解,认为改革开放是在
        走资本主义路线,改革开放的推行收到一定阻碍,但是随着邓小平同志的南方讲话和一些列座谈会的
        召开,人们逐渐加深了什么是社会主义、怎么建设社会主义的认识,开始探索出了中国特色社会主义的道路
        ,将改革开放定为基本国策,在改革开放政策的指引下,中国的经济实现了飞速发展,人民生活水平
        得到了极大的提高。\par
        在个人成长过程中,总会遇到失败,从失败中获取教训,总结经验,才成收获成长。成长的过程就是
        新旧事物交替运动的过程。在失败中总结经验,反思学习,是经验和思想“扬弃”的过程,我们在反思中
        获得新的感悟和体会,对一件事物的发展规律有了新的认识与看法,并且在这个新的体会下指导我们接下
        来的生活实践,这样我们就收获了成长。\par
    \subsection{辩证矛盾观指导下的生产生活活动}

    矛盾存在于一切事物之中,矛盾存在一切事物发展的始终。列宁将矛盾的原理称作辩证法的本质。
    矛盾是事物的普遍本质。矛盾双方的相互运动是事物发展的动力。矛盾分析法是
    解决矛盾的根本方法。把握事物内部的矛盾运动规律,可以帮助我们解决现实问题。
    主要矛盾是各种具体矛盾之中,影响范围最大,拥有
    支配地位的矛盾。它既有普遍性,又有特殊性。在生产生活中,学习和掌握好
    矛盾分析法,能对我们生产生活中困难的解决提供莫大的帮助。
        \subsubsection{矛盾具有普遍性和特殊性}
        矛盾的普遍性和特殊性的规律指出,任何事物在发展的任何阶段都有自己的矛盾,
        同类事物相同阶段的矛盾又具有相似性,这便是矛盾的普遍性。而不同事物的
        矛盾总归不是同一个矛盾,不同的矛盾有不同的特点,同一事物发展的不同阶段也
        有不同的矛盾,这便是矛盾的特殊性。\par
        矛盾的普遍性和特殊性原理告诉我们,要敢于承认矛盾,勇于揭露矛盾,
        努力解决矛盾。“不同质的矛盾,只有用不同质的方法才能解决。”\cite{毛泽东}
        要坚持具体问题具体分析。\par
        就国家和社会来说,不同的社会有不同的基本矛盾。党的十九大指出,现阶段
        我国的主要矛盾发生了变化,这是根据当前我国的国情做出的科学论断,在新的
        主要矛盾的背景下,党中央科学制定适合我国现阶段国情的发展规划,做到了具体问题
        具体分析,这便是在矛盾普遍性和特殊性原理指导下的生动实践。\par
        就个人来说,人在不同的阶段有不同的矛盾。婴幼儿时期有学说话、学走路的矛盾,
        学生时期有学习知识、增长才干的矛盾,成年后有照顾家庭、努力工作的矛盾、年老后有
        防病治病、保持健康的矛盾……在不同的阶段面临的主要矛盾不同,也应该有不同的解决方式。
        \subsubsection{主要矛盾和矛盾的主要方面}
        “在复杂的事物的发展过程中,有许多的矛盾存在,其中必有一种是主要的矛盾,
        由于它的存在和发展规定或影响着其他矛盾的存在和发展。”\cite{毛泽东}主要矛盾的
        主要方面决定着事物的性质。\par
        如在抗日战争时期,中日民族矛盾上升社会主要矛盾,阶级矛盾成为次要矛盾。
        而中日民族矛盾中,由于中国人民同仇敌忾,万众一心,中方在战争后期掌握战争主动权,
        取得抗日战争的胜利成为中日民族矛盾的主要方面。而到了解放战争时期,阶级矛盾成为主要矛盾,
        在这样主要矛盾的背景下,我们取得了解放战争的胜利,建立了新中国。\par
        可以说,正是由于一个个矛盾的解决,新中国才能建立,人民的生活水平才能越来越高。





    \subsection{辩证联系观指导下的生产生活活动}
    唯物辩证法指出,世界是普遍联系的。联系是客观的,是不以人的意志为转移的。
    世界上的事物都在与周围的事物产生者这样或者那样的联系,绝对孤立的事物是不存在
    的。我们需要了解辩证联系观,在联系观的指导下开展生产生活实践。
        \subsubsection{联系是客观的,多样的}
        联系是不以人的意志为转移的,是客观存在的。事物之间在不同的条件下有这样或者那样的
        联系,有本质的联系,也有表面的联系;有偶然的联系,也有必然的联系;有短期的联系,
        也有长期的联系……我们要把握好事物之间各种各样的联系,利用联系来方便自己的实践活动
        ,切不可割裂事物之间的联系盲目蛮干。要注意分析联系存在的各种条件,一切以时间、地点
        、条件为转移。
        \subsubsection{质量互变规律}
        量变是质变的必要准备,质变是量变的必要结果。我们希望事物有好的质变,就必须从好的量变
        开始做起。急于求成是要不得的,“一口吃不成胖子”说的就是这个道理。其实,生活中很多俗语
        已经为我们阐明了这个道理:
        “合抱之木,生于毫末;九层之台,起于累土;千里之行,始于足下”
        启示我们要任何成就都不是一蹴而就的,都是一点一滴做出来的。
        “千里之堤,溃于蚁穴”告诫我们要防微杜渐,坚决遏制坏的量变。
        唯物辩证法将这些谚语中蕴含的道理揭示了出来,应该在辩证法的指导下科学合理地安排
        生产实践,不可脱离实际,切忌“假、大、空”,时刻重视量的积累,为质变的发生做足准备。
        \newpage
        \addcontentsline{toc}{section}{参考文献}
        \bibliographystyle{plain}
        \bibliography{mybib.bib}       
      
\end{document}